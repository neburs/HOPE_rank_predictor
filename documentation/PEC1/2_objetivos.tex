\chapter{Objetivos del Máster}
\label{chapter:objetivos}


\section{Objetivo principal}

\label{op:OP1}
\paragraph{OP} - Poder recomendar al profesional sanitario las referencias bibliográficas actuales más exactas útiles y personalizadas que sean posibles, que pueden ayudar en el tratamiento del paciente, en base a la información que se dispone de este, pudiendo realizar una clasificación (\textit{ranking}) de más interés a menos.

\section{Objetivos secundarios}

\paragraph{}
Para poder cumplir con el objetivo principal \hyperref[op:OP1]{OP1}, desglosaremos los siguientes objetivos secundarios:

\label{os:OS1}
\paragraph{OS1} - Extraer la información de la base de datos y tratarla para quedarnos solo con la que consideramos valida.

\label{os:OS2}
\paragraph{OS2} - Hacer un análisis de componentes principales (estudio de que atributos son relevantes para alcanzar el objetivo).

\label{os:OS3}
\paragraph{OS3} - Enriquecer de los datos (\textit{data augmentation}) prediciendo los resultados que no están indicados si son relevantes o no. Aproximación por Vecinos más próximos (\textit{K-Nearest-Neighbor}).

\label{os:OS4}
\paragraph{OS4} - Predecir los resultados usando el algoritmo de aprendizaje supervisado para clasificación llamado Regresión logística '\textit{Logistic regression}'.

\label{os:OS5}
\paragraph{OS5} - Predecir los resultados usando el algoritmo de aprendizaje supervisado para clasificación llamado Bosques Aleatorios '\textit{Random Forests}'.

\label{os:OS6}
\paragraph{OS6} - Predecir los resultados usando el algoritmo de aprendizaje supervisado para clasificación llamado Máquinas de vector soporte '\textit{Support Vector Machines}'.

\label{os:OS7}
\paragraph{OS7} - Comparar los resultados obtenidos de los 3 modelos.