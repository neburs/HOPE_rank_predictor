\chapter{Conclusiones}
\label{chapter:conclusiones}

\section{Objetivos Secundarios}

\paragraph{}
En el apartado \nameref{section:analisis_datos} \textbf{se ha cumplido} el \hyperref[os:OS1]{OS1}, aplicado las transformaciones necesarias a los datos que tenemos para poder trabajar posteriormente con ellos con los modelos predictivos.

\paragraph{}
En el apartado \nameref{section:pca} \textbf{se ha cumplido} el \hyperref[os:OS2]{OS2}, pudiendo observar en el apartado de \hyperref[resultados:pca]{resultados}, que hay mucha diferencia según si se usa todas las observaciones o solo las que tienen informado el atributo a predecir (\textit{utilidad}), por lo que se decide junto con el cliente, intentar enriquecer los datos que no tienen informado el atributo (\textit{utilidad}) definiendo este paso como objetivo secundario \hyperref[os:OS3]{OS3}. Se analizara en el siguiente paso si el resultado de intentar enriquecer estos datos aporta algo al conjunto de datos.

\paragraph{}
En el apartado \nameref{section:knn} \textbf{no se ha podido cumplir} el \hyperref[os:OS3]{OS3}, pudiendo observar en el apartado de \hyperref[resultados:knn_conclusiones]{resultados} que el dataset se encuentra \hyperref[resultados:knn]{sesgado}, por lo que intentar enriquecer los datos lo único que generaría son falsos positivos, perjudicando a los posteriores entrenamientos de modelos predictivos que se estudian en este documento.

\paragraph{}
En el apartado \nameref{section:lr} \textbf{se ha cumplido} el \hyperref[os:OS4]{OS4}, realizando el estudio de la aplicación del modelo de Regresión Logística, pudiendo observar su \hyperref[resultados:lr]{bajo nivel de precisión}, llegando a desaconsejar su uso con el conjunto de datos actual.