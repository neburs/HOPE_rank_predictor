\section{Objetivo Principal}

\paragraph{}
Después de analizar los resultados obtenidos al entrenar los 3 modelos podemos concluir que, para predecir los resultados basándonos en los datos que tenemos actualmente, el \textbf{modelo que mejor precisión da} es el \hyperref[table:comparative]{\textbf{\textit{Bosques aleatorios}}}.

\paragraph{}
Para poder \textbf{predecir nuevos resultados}, sera necesario \textbf{aplicar} a estos, \textbf{las transformaciones que se han aplicado en el caso de estudio 2} (desde el formato de dato original, añadiendo el mes y año del artículo, eliminando los atributos \textit{gender} y artículo y expandiendo el atributo \textit{respuesta.pubmed\_keys}).

\paragraph{}
Finalmente podemos afirmar que \textbf{hemos cumplido parcialmente el objetivo principal \hyperref[op:OP1]{OP1}}, ya que con el modelo entrenado, seremos capaces de poder recomendar al profesional sanitario las referencias bibliográficas actuales útiles y personalizadas, que refuerce las tomas de decisiones en base a la información que se dispone de este. Lo que no podemos hacer debido a las características de los datos es realizar una clasificación (\textit{ranking}) de más interés a menos, como comentamos en el apartado \hyperref[section:limit]{Limitaciones detectadas}.

\section{Trabajos de futuro}

\paragraph{}
Existen varias lineas por las que seguir avanzando con la idea de poder mejorar los resultados y que se ajusten a las necesidades de información que requieren los profesionales sanitarios. Esto dependera de si se quiere continuar con la idea de solo indicar si un articulo es util o no o si por el contrario se quiere realizar un ranking valorando la utilidad de ese articulo. A continuación comentamos algunas lineas de futuro en base a estas dos opciones:


\subsection{Mejorar el modelo actual}
\paragraph{•} Mejorar el actual modelo de Bosques aleatorios añadiendo más observaciones para mejorar la prediccion de este. A medida que 


\subsection{Crear un modelo capaz de realizar un ranking de utilidad}
TODO