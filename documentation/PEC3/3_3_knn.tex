\section{Enriquecimiento de los datos}
\label{section:knn}

\paragraph{}
Debido a que tenemos un conjunto de datos con poco volumen, se acuerda el intentar enriquecer los datos (\hyperref[os:OS3]{OS3}) que no tienen un valor '\textit{utilidad}' definido. Como vimos en el apartado de \hyperref[section:analisis_datos_utilidad]{análisis de datos}, en el conjunto de datos aparecen muchas observaciones que no tienen indicado el atributo '\textit{utilidad}', ya que todavía no han sido valorados por los profesionales sanitarios. Estas observaciones se podrían rellenar intentando aproximarlas teniendo en cuenta valores de observaciones con similares características. En los siguientes apartados valoraremos si el resultado de esas aproximaciones seria lo suficientemente correcto para acabar aplicándolo.

\paragraph{}
Para realizar esta operación utilizaremos el algoritmo de Enrquicecimiento por Aproximación de Vecinos más próximos (conocido por K-Nearest-Neighbor o K-NN). Este Algoritmo pretende asociar si un registro pertenece a un conjunto de datos o a otro dependiendo de la aproximación de otros resultados de los que si se conoce su valor \cite{ref:knn_def}.

\paragraph{}
Para poder ejecutar el algoritmo de K-NN necesitamos realizar la transformación de todos los
atributos Categóricos (texto) a Continuos (números continuos) y estandarizar, igual que se realizo para el caso del \hyperref[section:pca_standar]{PCA}. Esto es debido a que el algoritmo necesita tener valores numéricos para poder trabajar con los datos\cite{ref:knn_scaling}. Con esta transformación ya podemos realizar operaciones matemáticas con el conjunto de datos.

\paragraph{}
Para realizar el K-NN se han seguido 2 estrategias, para valorar que impacto tiene las observaciones según el conjunto de datos a analizar. Hemos realizado 2 iteraciones con el siguiente conjunto de datos:

\paragraph{• 1: } El dataset completo con las transformaciones mencionadas en el apartado \nameref{section:analisis_datos} eliminando las observaciones que no tenían identificado el atributo edad.

\paragraph{• 2: } El mismo dataset realizado en la iteración 1 pero eliminando los atributos \textit{gender} (ya que todas las observaciones tienen el mismo valor) y artículo (que solo representa un identificador de un artículo). Ademas se expande el atributo \textit{respuesta.pubmed\_keys} para que exista una observación por cada keyword.

\paragraph{}
Una vez realizado esa transformación ya podemos ejecutar el K-NN. Todos los pasos para realizar el K-NN se pueden consultar en el anexo (\nameref{anx03:knn1}) para la iteración 1 y el anexo (\nameref{anx03:knn2}) para la iteración 2.