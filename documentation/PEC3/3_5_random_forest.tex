\section{Bosques Aleatorios '\textit{Random Forests}'}
\label{section:rf}

\paragraph{}
Otro modelo que escogemos como posible para intentar resolver el problema dado el atributo a predecir, es el modelo de aprendizaje supervisado para clasificación conocido como Bosques Aleatorios '\textit{Random Forests}'\cite{ref:rf_def} \hyperref[os:OS5]{OS5}. Escogemos este como posible modelo de confianza a analizar, debido a que su algoritmo suele tener un porcentaje de acierto elevado en conjuntos de datos con muchos atributos.

\paragraph{}
Igual que para el caso del \hyperref[section:knn]{K-NN}, realizaremos la transformación de todos los atributos Categóricos (texto) a Continuos (números continuos) y estandarizaremos los valores (\textbf{Nota}: Este modelo no es necesario estandarizar los valores pero he decidido hacerlo para igualar los valores de los atributos de entrada para todos los modelos).

\paragraph{}
Para ejecutar el modelo de Bosques Aleatorios se han seguido las mismas \hyperref[section:lr_casos]{2 estrategias} que para el caso del modelo de Regresión Logística, para así poder comparar los resultados de este modelo con el anterior.

\paragraph{}
Todos los pasos para realizar los Bosques Aleatorios se pueden consultar en el anexo (\nameref{anx05:rf1}) para la iteración 1 y el anexo (\nameref{anx05:rf2}) para la iteración 2.
