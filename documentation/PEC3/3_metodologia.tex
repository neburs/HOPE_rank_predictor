\chapter{Metodología}
\label{chapter:metodologia}


\section{Análisis del contexto}

\paragraph{}
Para poder comprender y abordar con éxito el \hyperref[os:OS1]{OS1} se ha realizado dos reuniones en donde el cliente expuso el origen de los datos para poder realizar el estudio.

\paragraph{}
En estas reuniones se pudo observar que los datos facilitados por el usuario requerían de una limpieza y tratamiento, ya que muchas observaciones tenían información poco relevante que podía generar ruido.

\section{Limitaciones detectadas}

\paragraph{}
Realizando un primer análisis visual, se detecto que los datos aportados por el cliente eran insuficientes para completar el \hyperref[op:OP1]{OP1}, ya que solo se disponía de la información respecto de si una referencia bibliográfica había sido útil o no, pero no se disponía de la información suficientemente detallada con información del grado de utilidad que había tenido esa referencia para poder llegar a realizar una clasificación (\textit{ranking}). El cliente nos comenta que en el momento actual no dispone de ese nivel de detalle.

\paragraph{}
Se acuerda con el cliente que intentara conseguir más volumen de información y con más detalle para poder realizar la clasificación (\textit{ranking}). Se comenta que se realizara una aproximación para indicar si una referencia bibliográfica es valida, dejando para mas adelante la opción de poder realizar la clasificación (\textit{ranking}) si se consigue ese nivel de detalle por parte del cliente.

\paragraph{}
También se pudo comprobar que el cliente disponía de un volumen de observaciones bajo por lo que se planteo la posibilidad de, o intentar obtener más observaciones, o enriquecer las actuales generando nuevos datos por aproximación a los reales.

\paragraph{}
Finalmente se decidió estudiar si era viable generar nuevos valores por aproximación, debido a que en el momento en que se trató el problema, el cliente no podía facilitar más datos. Si a lo largo del estudio, conseguía obtener nuevas observaciones, estas serian añadidas al estudio para aproximar mejor la solución final.

\section{Análisis de datos}
\label{section:analisis_datos}

\paragraph{}
Tras recibir los datos por parte del cliente y realizar el primer análisis superficial, se detecto que el volumen de datos no era suficiente como para que los resultados que pudieran surgir del estudio, fueran concluyentes. Todo y con eso se acordó (tal y como se refleja en el \hyperref[os:OS1]{OS1}) hacer una valoración de los datos para ver si se podrían enriquecer de algún modo, mientras el cliente intenta obtener más volumen de datos.

\paragraph{}
Para ello, transformamos los datos que se nos facilito desde una base de datos donde se encontraba la información guardada en formato json, a un formato columnar en el que poder aplicar los modelos predictivos. Una vez tenemos los datos transformados, observamos que existen ciertos atributos que contienen listas de opciones como son los atributos \textit{pubmed\_keys} (que corresponde a las palabras clave que la API\cite{ref:pubmed_api} de pubmed nos devuelve para esta observación), \textit{articles} (que corresponde a los ids de los artículos relacionados con esa observación), \textit{articlesRevisedYear} y \textit{articlesRevisedMonth} (que corresponde a los años y meses de los artículos según están ordenados en el atributo \textit{articles}). Para poder recomendar artículos útiles, necesitamos tener una observación por artículo, para poder analizar posteriormente de manera independiente si ese artículo fue útil o no para la observación a la que hace referencia, por lo que aplicaremos las transformaciones necesarias para acabar obteniendo la representación de un artículo con su correspondiente \textit{feedback} por parte de los profesionales sanitarios. Todos los pasos para realizar las transformaciones se pueden consultar en el anexo (\nameref{anx01:procesado_datos}).

\paragraph{}
Aplicados las transformaciones anteriores, nos quedan los siguientes atributos que describimos a continuación:

\paragraph{• pedido.data.attributes.age:} Nos indica la edad del paciente en formato numérico (ejemplos de valores $=>$ 75,86,40,...).
\paragraph{• pedido.data.attributes.diagnostic\_main:} Nos indica el diagnostico principal que dio el profesional sanitario para la enfermedad que tiene el paciente en formato texto (ejemplos de valores $=>$ \textit{Fistula Peritoneal, Insuficiencia Respiratoria},...).
\paragraph{• pedido.data.attributes.gender:} Nos indica el sexo del paciente en formato texto (ejemplo de valor $=>$ \textit{male}).
\paragraph{• artículo:} Nos indica el identificador del artículo relacionado con el diagnostico principal en formato numérico (ejemplos de valores $=>$ 28694230,28805236,...).
\paragraph{• respuesta.articlesRevisedYear:} Nos indica el año de revisión del artículo referenciado por el identificador del campo artículo en formato numérico (ejemplos de valores $=>$ 2018,2017,2016,...).
\paragraph{• respuesta.articlesRevisedMonth:} Nos indica el mes de revisión del artículo referenciado por el identificador del campo artículo en formato numérico (ejemplos de valores $=>$ 4,12,6,9,...).
\paragraph{• respuesta.pubmed\_keys:} Nos indica las palabras clave relacionadas con el artículo del campo artículo en formato texto (ejemplos de valores $=>$ \textit{'Abdomen, Adenocarcinoma, Antiemetics, Blood', 'Abdomen, Analgesics, Bone, Catharsis', 'Abdomen, Anti-Bacterial Agents, Diuresis'},...).
\label{section:analisis_datos_utilidad}
\paragraph{• utilidad:} Nos indica si un profesional sanitario ha considerado si el artículo referenciado en el campo artículo es útil para el diagnostico principal relacionado con la enfermedad del paciente. Este campo puede tener 3 valores: 1 para saber que el artículo es útil, 0 para indicar que no es útil y null para indicar que aun no se ha valorado.
\section{Análisis de componentes principales}
\label{section:pca}

\paragraph{}
El análisis de componentes principales (o como se le conoce en ingles por '\textit{Principal Component Analysis}' o \textit{PCA}), es un componente fundamental en el análisis de los datos, ya que permite reducir el número de atributos de un conjunto de datos para eliminar el ruido que los posteriores análisis/modelos predictivos funcionen con mejor precisión\cite{ref:pca_def}.

\paragraph{}
Por poner un ejemplo sencillo, imaginemos que tenemos un conjunto de datos de modelos de coche con muchísimos atributos de estos, como por ejemplo, el nombre del modelo, el color, el número de puertas, la cilindrada y la potencia, entre otros. Probablemente si intentáramos analizar los datos en busca de cual es el modelos que menos consume y quisiéramos crear un modelo predictivo con este objetivo, podríamos observar a simple vista que tenemos atributos que no nos son necesarios y que generaría distracción (ruido) a la hora de conseguir nuestro objetivo, como es el caso del atributo color, o el nombre del modelo.

\paragraph{}
El \textit{PCA} nos ayudara a encontrar cuales son los atributos más significativos del conjunto de datos para conseguir predecir el atributo que queremos\cite{ref:pca_def}, que en el caso que nos toca, es el atributo '\textit{utilidad}'.

\paragraph{}
Es cierto que, en nuestro conjunto de datos, no tenemos un gran volumen de atributos, pero este proceso nos puede ayudar a eliminar atributos con poca relevancia, para así, poder simplificar el modelo predictivo final.

\label{section:pca_standar}
\paragraph{}
Para realizar el \textit{PCA} nos ayudaremos de la librería \textit{sklearn.decomposition} que ya nos ofrece implementada la lógica para ejecutarlo. Ademas realizaremos la transformación de todos los atributos de Categóricos (texto) a Continuos (números continuos). Este paso se realiza para que el \textit{PCA} pueda realizar operaciones matemáticas sobre los valores de las observaciones. A este proceso se le conoce como factorización (\textit{factorize}). 

\paragraph{}
También estandarizaremos los valores a un rango de entre 1 y -1. Esto se realiza para igualar la importancia de todos los atributos, ya que en el paso anterior, al realizar la factorización, se nos puede dar el caso de tener valores muy altos (por ejemplo, al factorizar un atributo con 100 valores diferentes, se nos dará casos de observaciones que en un atributo tienen el valor 100, que puede ser más alto que otros valores que no se han transformado). Al realizar el PCA, si no se hace esta estandarización de los datos, los valores más altos se les dará más peso, pero no por eso pueden ser relevantes. Por lo que es imperativo el realizar esta estandarización.

\paragraph{}
Para realizar el PCA se han seguido 3 estrategias, para valorar que impacto tiene las observaciones según el conjunto de datos a analizar. Hemos realizado 3 iteraciones con el siguiente conjunto de datos:

\paragraph{• 1: } El dataset completo con las transformaciones mencionadas en el apartado \nameref{section:analisis_datos}

\paragraph{• 2: } El mismo dataset de la anterior iteración pero cogiendo solo las observaciones que se ha informado el atributo utilidad.

\paragraph{• 3: } El mismo dataset realizado en la iteración 2 pero añadiendo el mes y año del articulo, eliminando los atributos \textit{gender} (ya que todas las observaciones tienen el mismo valor) y articulo (que solo representa un identificador de un articulo). Ademas se expande el atributo \textit{respuesta.pubmed\_keys} para que exista una observación por cada keyword.

\paragraph{}
Una vez realizado esa transformación ya podemos ejecutar el PCA. Todos los pasos para realizar el PCA se pueden consultar en el anexo (\nameref{anx02:pca1}) para la iteración 1, el anexo (\nameref{anx02:pca2}) para la iteración 2 y el anexo (\nameref{anx02:pca3}) para la iteración 3.
\section{Enriquecimiento de los datos.}
\label{section:knn}

\paragraph{}
Debido a que tenemos un conjunto de datos con poco volumen, se acuerda el intentar enriquecer los datos (\hyperref[os:OS3]{OS3}) con los conjuntos de datos que tenemos que no tienen informado el atributo a predecir.

\paragraph{}
Para realizar esta operación utilizaremos el algoritmo de Enrquicecimiento por Aproximación de Vecinos más próximos (conocido por K-Nearest-Neighbor o K-NN). Este Algoritmo pretende asociar si un registro pertenece a un conjunto de datos o a otro dependiendo de la aproximación de otros resultados de los que si se conoce su valor \cite{ref:knn_def}.

\paragraph{}
Para poder ejecutar el algoritmo de K-NN necesitamos realizar la transformación de todos los
atributos Categóricos (texto) a Continuos (números continuos) y estandarizar, igual que se realizo para el caso del \hyperref[section:pca_standar]{PCA}. Esto es debido a que el algoritmo necesita tener valores numéricos para poder trabajar con los datos\cite{ref:knn_scaling}. Con esta transformación ya podemos realizar operaciones matemáticas con el conjunto de datos.

\paragraph{}
Para realizar el K-NN se han seguido 2 estrategias, para valorar que impacto tiene las observaciones según el conjunto de datos a analizar. Hemos realizado 2 iteraciones con el siguiente conjunto de datos:

\paragraph{• 1: } El dataset completo con las transformaciones mencionadas en el apartado \nameref{section:analisis_datos} eliminando las observaciones que no tenían identificado el atributo edad.

\paragraph{• 2: } El mismo dataset realizado en la iteración 1 pero eliminando los atributos \textit{gender} (ya que todas las observaciones tienen el mismo valor) y artículo (que solo representa un identificador de un artículo). Ademas se expande el atributo \textit{respuesta.pubmed\_keys} para que exista una observación por cada keyword.

\paragraph{}
Una vez realizado esa transformación ya podemos ejecutar el K-NN. Todos los pasos para realizar el K-NN se pueden consultar en el anexo (\nameref{anx03:knn1}) para la iteración 1 y el anexo (\nameref{anx03:knn2}) para la iteración 2.
\section{Regresión Logística}
\label{section:lr}

\paragraph{}
Una vez analizado el conjunto de datos aportado por el cliente, podemos ver que finalmente el atributo a predecir \textit{utilidad} es de tipo booleano (0 haciendo referencia a los documentos con poco o nada de interés para los profesionales sanitarios y 1 para los que si son de interés).

\paragraph{}
Como el problema a resolver es de clasificación, escogemos como posible modelo de aprendizaje supervisado para clasificación a analizar el conocido como Regresión Logística '\textit{Logistic regression}'\cite{ref:lr_def} \hyperref[os:OS4]{OS4}. Escogemos este como posible modelo de confianza a analizar debido a que es un modelo muy sencillo de entrenar y comprender su comportamiento\cite{ref:lr_understanding}.

\paragraph{}
Igual que para el caso del \hyperref[section:knn]{K-NN}, realizaremos la transformación de todos los atributos Categóricos (texto) a Continuos (números continuos) y estandarizaremos los valores.

\paragraph{}
Para ejecutar el modelo de Regresión Logística se han seguido 2 estrategias, para valorar que impacto tiene las observaciones según el conjunto de datos a analizar. Hemos realizado 2 iteraciones con el siguiente conjunto de datos:

\label{section:lr_casos}
\paragraph{• 1: } El dataset completo con las transformaciones mencionadas en el apartado \nameref{section:analisis_datos} eliminando las observaciones que no tenían identificado el atributo edad y escogiendo los atributos que nos ha indicado como relevantes el \hyperref[result:pca_case2]{caso de estudio 2} del PCA.

\paragraph{• 2: } El mismo dataset realizado en la iteración 1 pero eliminando los atributos \textit{gender} (ya que todas las observaciones tienen el mismo valor) y artículo (que solo representa un identificador de un artículo). Ademas se expande el atributo \textit{respuesta.pubmed\_keys} para que exista una observación por cada keyword. También escogemos solo los atributos que nos ha indicado como relevantes el \hyperref[result:pca_case3]{caso de estudio 3} del PCA.

\paragraph{}
Todos los pasos para realizar la Regresión Logística se pueden consultar en el anexo (\nameref{anx04:rl1}) para la iteración 1 y el anexo (\nameref{anx04:rl2}) para la iteración 2.

\section{Bosques Aleatorios '\textit{Random Forests}'.}
\label{section:rf}

\paragraph{}
Otro modelo que escogemos como posible para intentar resolver el problema, dada la casuistica del dataset y el atributo a predecir, es el modelo de aprendizaje supervisado para clasificación conocido como Bosques Aleatorios '\textit{Random Forests}'\cite{ref:rf_def} \hyperref[os:OS5]{OS5}. Escogemos este como posible modelo de confianza a analizar, debido a que su algoritmo suele tener un porcentaje de acierto elevado en conjuntos de datos con muchos atributos.

\paragraph{}
Igual que para el caso del \hyperref[section:knn]{K-NN}, realizaremos la transformación de todos los atributos Categóricos (texto) a Continuos (números continuos) y estandarizaremos los valores (\textbf{Nota}: Este modelo no es necesario estandarizar los valores pero he decidido hacerlo para igualar los valores de los atributos de entrada para todos los modelos).

\paragraph{}
Para ejecutar el modelo de Bosques Aleatorios se han seguido las mismas \hyperref[section:lr_casos]{2 estrategias} que para el caso del modelo de Regresión Logística, para así poder comparar los resultados de este modelo con el anterior.

\paragraph{}
Todos los pasos para realizar los Bosques Aleatorios se pueden consultar en el anexo (\nameref{anx05:rf1}) para la iteración 1 y el anexo (\nameref{anx05:rf2}) para la iteración 2.
