\section{Bosques Aleatorios '\textit{Random Forests}'.}
\label{resultados:rf}

\subsection{Características de los conjuntos de datos a analizar}
\label{resultados:rf_caracteristicas}
Al realizar las mismas transformaciones que para el caso de \hyperref[resultados:knn_caracteristicas]{K-NN}, las características de los conjuntos de datos son las mismas.

\paragraph{}
EXPLICAR EL ENTRENAMIENTO DEL MODELO

\subsection{(Caso de estudio 1) Resultados del entrenamiento con el Conjunto de datos solo con atributo utilidad definido y escogiendo los atributos que nos ha indicado como relevantes el \hyperref[result:pca_case2]{caso de estudio 2} del PCA}

TODO

\subsection{(Caso de estudio 2) Conjunto de datos solo con atributo utilidad definido, añadiendo el mes y año del artículo, eliminando los atributos \textit{gender} y artículo y expandiendo el atributo \textit{respuesta.pubmed\_keys}. También escogemos solo los atributos que nos ha indicado como relevantes el \hyperref[result:pca_case3]{caso de estudio 3} del PCA.}

TODO

\subsection{Conclusiones del modelo de Regresión Logística}
\label{resultados:rf_conclusiones}

TODO
