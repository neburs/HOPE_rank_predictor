\chapter{Introducción}
\label{chapter:introduccion}


%%% SECTION
\section{Descripción general del problema}
\label{def:def1}

\paragraph{}
El Trabajo de Final de Máster (\textit{TFM}) que aquí se presenta nace de la necesidad por parte de los profesionales sanitarios de poder tener la información más exacta posible sobre las mejores referencias bibliográficas actuales sobre tratamientos a aplicar a un paciente, dado unos síntomas concretos. Actualmente existe infinidad de referencias medicas que los profesionales sanitarios pueden consultar, pero esta información es tan abundante que acaba siendo engorrosa de consultar\cite{ref:search_internet}. Esto hace que, muchas veces sea complicado encontrar la información sobre estas referencias bibliográficas para tratar algunas enfermedades. En el ámbito de la medicina el tiempo perdido puede costar vidas y es un precio demasiado elevado a pagar, tanto a nivel económico como emocional.

\paragraph{}
Los profesionales sanitarios necesitan poder disponer de una plataforma que sea capaz de poder facilitarles estas referencias bibliográficas actuales lo más exactas y personalizadas que sean posible, ajustándose a la búsqueda de la información que poseen de sus pacientes.

\paragraph{}
Bajo esa premisa nace el proyecto HOPE\cite{ref:hope_home} (que significa \textit{Health Operations for Personalized Evidence} en ingles) con el objetivo de ayudar a estos profesionales sanitarios a encontrar estas referencias bibliográficas que necesitan de la manera más rápida y fácil posible. Actualmente existen bases de datos de confianza en donde los profesionales sanitarios y el publico en general puede buscar estas referencias de ensayos y estudios clínicos con información fiable y desarrollados con anterioridad, pero no siempre es fácil o rápido encontrar estos resultados\cite{ref:search_results_study}.

\paragraph{}
El proyecto HOPE es un sistema basado en inteligencia artificial para identificar la información de casos clínicos registrados en la Historia Clínica Electrónica, en base a los cuales realiza una búsqueda única por paciente para proporcionar al profesional sanitario recomendaciones de referencias bibliográficas donde constan tratamientos, estudios de investigación e información para ayudar al paciente. Todo en base a registros de fuentes científicas de información. En este proyecto, los profesionales sanitarios de todo el mundo puede consultar en una base de datos estas referencias y ver que otros tratamientos relacionados con los síntomas de sus pacientes han dado resultado. Todo y con eso, el sistema no siempre devuelve las referencias actuales más relevantes por lo que, no siempre la información consultada es útil.

\paragraph{}
Ademas, actualmente los profesionales sanitarios pueden valorar si la información recibida ha sido útil o no respecto a la búsqueda que han realizado, por lo que con esta valoración (\textit{feedback}), se pretende mejorar el sistema actual complementándolo con un modelo clasificador capaz de ayudar al actual a entregar realmente las referencias que son más útiles basándose en la valoración que los profesionales sanitarios dan al sistema.

\paragraph{}
Este Trabajo de final de máster (\textit{TFM}) pretende ayudar al proyecto HOPE a mejorar su algoritmo de Inteligencia Artificial para que los resultados se ajusten a las necesidades de información que requieren los profesionales sanitarios, en base a las búsquedas personalizadas que puedan hacer respecto a la información que tienen de sus pacientes. Para hacer esto se realizara un estudio de aproximación a conocer cual es el mejor modelo predictivo que puede ayudar a devolver esa información lo más exacta posible.

\section{Motivación personal}

\paragraph{}
En los duros tiempos en los que estamos viviendo actualmente, tanto económica como emocionalmente, es grato ver como la humanidad es capaz de dejar a un lado sus diferencias y unirse para afrontar problemas comunes. En el caso del proyecto HOPE, lo que más me atrajo fue la oportunidad de poder ayudar a encontrar soluciones a enfermedades que ya están en el último paso (o como médicamente se le describe, en cuidados paliativos).

\paragraph{}
Es evidente que este proyecto no va a dar una solución para curar cualquier enfermedad, pero si puede ayudar a los profesionales sanitarios a poder encontrar posibles soluciones a enfermedades complejas, ayudando a estos profesionales a buscar en la infinidad de documentación medica que existe, el tratamiento que más pueda ayudar a paliar o, quien sabe, curar una enfermedad que ya se daba por incurable. Esto mismo es lo que me motiva y mucho el poder ayudar ante estas situaciones.

\paragraph{}
También me ha motivado muchísimo el conocer a gente profesional que, independientemente del país al que pertenece o la profesión que tiene, se una al proyecto HOPE para participar y ayudar con sus conocimientos a hacer de este mundo, un lugar un poco mejor en el que vivir. Esta experiencia me esta enriqueciendo muchísimo personalmente y espero poder seguir contribuyendo al proyecto cuando este máster acabe.
