\chapter{Metodología}
\label{chapter:metodologia}


\section{Reuniones con el cliente}

\paragraph{}
Para poder comprender y abordar con éxito el \hyperref[op:OP1]{objetivo principal} se ha realizado dos reuniones en donde el cliente expuso el \hyperref[op:OP1]{problema} a abordar y el origen de los datos para poder realizar el estudio.

\paragraph{}
En estas reuniones se pudo observar que los datos facilitados por el usuario requerían de una limpieza y tratamiento para poder cumplir el objetivo principal, ya que muchas observaciones tenían información poco relevante que podía generar ruido.

\section{Limitaciones detectadas}

\paragraph{}
Realizando un primer análisis visual, se detecto que los datos aportados por el cliente eran insuficientes para completar el \hyperref[op:OP1]{OP1}, ya que solo se disponía de la información respecto de si una referencia bibliográfica había sido útil o no, pero no se disponía de la información suficientemente detallada para saber si había sido muy útil o poco útil para poder llegar a realizar una clasificación (\textit{ranking}). El cliente nos comenta que en el momento actual no dispone de ese nivel de detalle.

\paragraph{}
Se acuerda con el cliente que intentara conseguir más volumen de información y con más detalle para poder realizar la clasificación (\textit{ranking}) que necesita. Todo y con eso, se comenta que se realizara una aproximación para indicar si una referencia bibliográfica es útil o no dejando para mas adelante la opción de poder realizar la clasificación (\textit{ranking}) si se consigue ese nivel de detalle por parte del cliente.

\paragraph{}
También se pudo comprobar que el cliente disponía de un volumen de observaciones bajo por lo que se planteo la posibilidad de, o intentar obtener más observaciones facilitadas por el cliente, o intentar enriquecer las observaciones actuales generando nuevos datos por aproximación a los reales.

\paragraph{}
Finalmente se decidió estudiar si era viable generar nuevos valores por aproximación, debido a que en el momento en que se trato el problema, el cliente no podía facilitar más datos. Si a lo largo del estudio, el cliente conseguía obtener nuevas observaciones, estas serian añadidas al estudio para aproximar mejor la solución final.
