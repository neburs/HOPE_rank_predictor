\pagenumbering{roman} 
\setcounter{page}{1} 
\pagestyle{plain}

%%%%%%%%%%%%%%%%
%%% CREDITOS %%%
%%%%%%%%%%%%%%%%
\chapter*{Créditos/Copyright}

\vspace{1cm}

\begin{figure}[ht]
    \centering
	\includegraphics[scale=1]{images/license.png}
\end{figure}

Esta obra está sujeta a una licencia de Reconocimiento -  NoComercial - SinObraDerivada

\href{https://creativecommons.org/licenses/by-nc-nd/3.0/es/}{3.0 España de CreativeCommons}.

%%%%%%%%%%%%%
%%% FICHA %%%
%%%%%%%%%%%%%
\chapter*{FICHA DEL TRABAJO FINAL}

\begin{table}[ht]
	\centering{}
	\renewcommand{\arraystretch}{2}
	\begin{tabular}{r | l}
		\hline
		Título del trabajo: & CLASIFICADOR DOCUMENTOS MÉDICOS HOPE\\
		\hline
        Nombre del autor: & Rubén Vasallo Gonzalez\\
		\hline
        Nombre del colaborador/a docente: & Carlos Luis Sanchez Bocanegra, Rafael Pastor Vargas \\
		\hline
        Nombre del PRA: & Jordi Casas Roma \\
		\hline
        Fecha de entrega (mm/aaaa): & 01/2021\\
		\hline
        Titulación o programa: & Máster Universitario en Ciencia de Datos\\
		\hline
        Área del Trabajo Final: & M2.979 - TFM\\
		\hline
        Idioma del trabajo: & Español\\
		\hline
        Palabras clave & hope, clasificador, medicina\\
		\hline
	\end{tabular}
\end{table}

%%%%%%%%%%%%%%%%%%%
%%% DEDICATORIA %%%
%%%%%%%%%%%%%%%%%%%
\chapter*{Dedicatoria/Cita}

Quiero dedicarle este trabajo a mis mentores \textit{Carlos Luis Sanchez Bocanegra} y \textit{Rafael Pastor Vargas} que me han ayudado, apoyado y guiado en todo momento para conseguir los objetivos del máster.

%%%%%%%%%%%%%%%%%%%
%%% Agradecimientos %%%
%%%%%%%%%%%%%%%%%%%
\chapter*{Agradecimientos}

\paragraph{}
Quiero agradecer a \textit{Carlos Luis Sanchez Bocanegra} por invitarme participar en el Proyecto HOPE y poder aportar mi granito de arena a este gran proyecto.

\paragraph{}
También quiero dar la gracias a todos los miembros del proyecto HOPE que me han dado la bienvenida al grupo y me han facilitado la vida en unas circunstancias en las que, cuando entre en este, no eran las más idóneas. La mayoría de integrantes del grupo son médicos y la saturación de trabajo que había por el COVID-19 era enorme. Tengo claro que sin ellos y sin la ayuda en especial de Carlos y \textit{Nicolas Passadore} que ha estado luchando para conseguir un dataset con más observaciones, este Máster no habría sido posible.

%%%%%%%%%%%%%%%%
%%% RESUMEN  %%%
%%%%%%%%%%%%%%%%
\chapter*{Abstract}
\addcontentsline{toc}{chapter}{Abstract}

\onehalfspacing
\paragraph{}
This Final Master's Thesis {\textit{FMT}} was born from the need to be able to have in a simple, up-to-date and immediate way, medical bibliographic references cataloged according to the information and the patient's symptoms, being able to make a ranking of more or less interest function of the feedback provided by health professionals on these bibliographic references.

\paragraph{}
\textbf{Resumen}:
\paragraph{}
Este Trabajo de Final de Máster (\textit{TFM}) nace de la necesidad de poder disponer de una manera sencilla, actualizada e inmediata, referencias bibliográficas médicas catalogadas según la información y los síntomas que tienen del paciente, pudiendo hacer una clasificación (\textit{ranking}) de más o menos interés en función de la valoración (\textit{feedback}) aportada por los profesionales sanitarios sobre estas referencias bibliográficas.

\vspace{1.5cm}

\textbf{Palabras clave}: clasificador, artículos, médicos, PCA, KNN, Regresión Logística, Random Forest, SVM